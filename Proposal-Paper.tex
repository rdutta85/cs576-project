%
\title{Chord Strike}
\author{
Jeffrey Deng
\and
Rishav Dutta
\and
Shuchi Mishra
\and
Brian Tan
\and
Trong Vo
}
\date{\today}
\documentclass[11pt]{article}
\usepackage{amsmath, amssymb, amsthm, amsfonts, mathtools, fullpage}
\begin{document}
\maketitle
\paragraph*{Game Description and Mechanics}
Chord Strike is an isomorphic 3D single-player adventure game using Unity Game Engine 
where players control a musician character who explores a fantastical world using music 
to defeat enemies and solve puzzles. The player navigates the environment using the arrow 
keys and plays music using the keys "a" through "f," each representing triad chords 
(e.g. C = C, E, G)s at the note of the letter key pressed. Pressing the space bar makes 
the triad minor, pressing the shift/control key sharpens/flattens the root note of the 
chord. Enemies in the game display musical note names above their heads, and the player
must play the corresponding chord to defeat them. When a chord is played, enemies within 
a cone-shaped area are damaged if their notes are part of the chord. The game also includes 
a final boss battle, where players must play a sequence of chords to create a progression 
in order to defeat it.

\paragraph*{Educational Component}
The goal of the game is to teach children and teens musical concepts and to develop an 
understanding of the basic chord structure and chord progressions. The game will include 
three distinct levels, each with unique music theory components being taught. Environmental 
puzzles require players to play specific chord sequences and scales to interact within the 
closed environment. 

\paragraph*{AI Component}
Enemy AI is based on Unity's NavMesh system, enabling enemies to chase the player and 
react to musical inputs. The game will also feature interactive environmental elements 
that respond to the player's music, such as doors and platforms that move when a specific 
chord is played.

\paragraph*{Animation and Sound Component}
The character's movement and chord-playing actions are accompanied by key-frame animations 
and sound effects. Playing a chord will trigger visual effects, such as sound waves or 
particles in a cone pattern to represent the range of effect. Enemies will react physically 
when defeated, "disintegrating" into musical notes (not dying). The boss character will also 
have animations, each reflecting its various phases in a chord progression. The game's sound 
design will be closely tied to the gameplay, with each chord generating its corresponding 
musical tones and the environment having a soundtrack that adapts based on player actions.

\paragraph*{UI Component}
The game will feature a simple user interface including a menu, level selection screens, and 
tutorials that teach music theory concepts. In-game visual cues will aim to assist players in 
learning the mechanics as they progress. The game will be designed primarily for keyboard 
controls on a computer. Players can unlock and switch between different instruments, each 
offering unique sounds and visual effects, and gameplay mechanics. 

\end{document}
